\section*{Introduzione}

Lo scopo principale di questi appunti è quello di esaminare più da vicino gli esami di programmazione funzionale tenuti all'Università degli Studi di Trento. %
Questi appunti non sono completi, e la loro lettura non permette, da sola, di superare l’esame. %
La versione più recente si trova all'indirizzo:

\begin{center}
	\url{github.com/emanuelenardi/latex-sml}
\end{center}

% \subsection*{Donazioni}
%
% Questa dispensa è in fase di riscrittura, sudore e lacrime sono stati versati.
% Per supportare l'autore in questo lungo e tortuoso viaggio effettua una \href{https://paypal.me/pools/c/85MUW0ex8l}{piccola donazione{\ExternalLink}}.

% \subsection*{Ringraziamenti}
%
% Un grazie di cuore a:
% \begin{itemize}
% 	\item Riccardo Persico che dice -- ``Grazie''
% 	\item Alessio Gandelli che dice -- ``grazie per tutto''
% \end{itemize}

\subsection*{Materiale}

Puoi trovare una veloce introduzione a Standard ML su \href{https://learnxinyminutes.com/docs/standard-ml/}{Learn X in Y minutes \ExternalLink}.

\smallskip
Ho prodotto una \href{bit.ly/sml-youtube-playlist}{playlist di youtube{\ExternalLink}} che tratta gli argomenti del corso.
Se trovi qualche video esplicativo e pensi che possa tornare utile ai tuoi compagni di corso, tramite questo link, puoi aggingerli direttamente alla playlist.

\smallskip
Per tutto il resto consulta la cartella \href{https://bit.ly/drive-folder}{Google Drive{\ExternalLink}} del corso triennale di Informatica.

\subsection*{Segnalazione di errori}

Se hai trovato un errore ti prego di aprire un \href{github.com/emanuelenardi/latex-sml/issues/new}{\textsf{issue}{\ExternalLink}} su \faicon{github} github.
