\section{Settembre 2017}

\subsection{Testo}

Si consideri il tipo di dato
\begin{lstlisting}[style = SML]
datatype intonil = Nil | Int of int
\end{lstlisting}

ed una possibile implementazione semplificata di ambiente (che considera solo valori interi) basata su di esso:
\begin{lstlisting}[style = SML]
type ambiente = string -> intonil
\end{lstlisting}

In questa implementazione, un ambiente è rappresentato da una funzione che mappa nomi (valori di tipo \sml{string}) in valori di tipo \sml{intonil} (che rappresentano un intero o nessun valore). Tale funzione applicata ad un nome ritorna il valore intero ad esso associato oppure \sml{Nil}.

\medskip
Usando questa convenzione, l'ambiente vuoto (in cui nessun nome è associato a valori) può essere definito come:
\begin{lstlisting}
val ambientevuoto = fn _:string => Nil;
\end{lstlisting}

Basandosi su queste definizioni, si definisca una funzione "\sml{lega}" con tipo "\sml{ambiente -> string -> int -> ambiente}". che a partire da un ambiente (primo argomento) genera un nuovo ambiente (valore di ritorno) uguale al primo argomento più un legame fra il nome e l'intero ricevuti come secondo e terzo argomento.

\medskip
\textbf{IMPORTANTE}: notare il tipo della funzione! Come si può intuire da tale tipo, usa la \emph{tecnica del currying} per gestire i suoi due argomenti.

\medskip
A titolo di esempio:

\begin{itemize}
  \item \sml{((lega ambientevuoto "a" 1) "a")} deve ritornare \sml{Int 1};
  \item \sml{((lega ambientevuoto "a" 1) "boh")} deve ritornare \sml{Nil};
  \item \sml{((lega (lega ambientevuoto "a" 1) "boh" ~1) "boh")} deve ritornare \sml{Int ~1};
  \item \sml{((lega (lega ambientevuoto "a" 1) "boh" ~1) "mah")} deve ritornare \sml{Nil}.
\end{itemize}

\subsection{Soluzione}

\begin{lstlisting}[style = SML, caption = {Definizione della funzione \sml{lega}}]
val lega = fn e:ambiente =>
	fn nome =>
		fn valore =>
			(fn n => if (n = nome)
			    then
					(Int valore)
			    else
					(e n)): ambiente;
\end{lstlisting}

\subsection{Commento della soluzione}

La \sml{e} indica simbolicamete l'\sml{ambiente}, mentre la \sml{n} il \sml{nome}.
