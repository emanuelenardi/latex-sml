\section{Settembre 2017}

\subsection{Testo d'esame}

Si consideri il tipo di dato

\begin{smlcode}
datatype intonil = Nil | Int of int
\end{smlcode}

ed una possibile implementazione semplificata di ambiente (che considera solo valori interi) basata su di esso:

\begin{smlcode}
type ambiente = string -> intonil
\end{smlcode}

In questa implementazione, un ambiente è rappresentato da una funzione che mappa nomi (valori di tipo \texttt{string}) in valori di tipo \texttt{intonil} (che rappresentano un intero o nessun valore). Tale funzione applicata ad un nome ritorna il valore intero ad esso associato oppure \texttt{Nil}.

\medskip
Usando questa convenzione, l'ambiente vuoto (in cui nessun nome è associato a valori) può essere definito come:

\begin{smlcode}
val ambientevuoto = fn _:string => Nil;
\end{smlcode}

Basandosi su queste definizioni, si definisca una funzione ``\texttt{lega}'' con tipo ``\texttt{ambiente -> string -> int -> ambiente}''. che a partire da un ambiente (primo argomento) genera un nuovo ambiente (valore di ritorno) uguale al primo argomento più un legame fra il nome e l'intero ricevuti come secondo e terzo argomento.

\medskip
\textbf{IMPORTANTE}: notare il tipo della funzione! Come si può intuire da tale tipo, usa la \emph{tecnica del currying} per gestire i suoi due argomenti.

\medskip
A titolo di esempio:
\begin{itemize}
	\item \texttt{((lega ambientevuoto "a" 1) "a")} deve ritornare \texttt{Int 1};
	\item \texttt{((lega ambientevuoto "a" 1) "boh")} deve ritornare \texttt{Nil};
	\item \texttt{((lega (lega ambientevuoto "a" 1) "boh" ~1) "boh")} deve ritornare \texttt{Int ~1};
	\item \texttt{((lega (lega ambientevuoto "a" 1) "boh" ~1) "mah")} deve ritornare \texttt{Nil}.
\end{itemize}

\subsection{Soluzione}

\begin{listing}{!h}
\smlfile{assets/codes/2017.09/lega.sml}
\caption[]{Definizione della funzione \texttt{lega}}
\end{listing}

\subsection{Commento della soluzione}

La \texttt{e} indica simbolicamete l'\texttt{ambiente}, mentre la \texttt{n} il \texttt{nome}.

\subsection*{Esempio di esecuzione}

TODO
