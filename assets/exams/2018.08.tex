\section{Agosto 2018}

\subsection*{Testo}

Si scriva una funzione \texttt{prod} che computi, dati come argomento due \texttt{int} \(n\) ed \(m\), il prodotto di tutti gli interi compresi tra \(m\) e \(n\). %
Qualora sia \(m \geqslant n\) la funzione ritona il valore di \(m\). %
Si impieghi inoltre questa funzione per definire la funzione \texttt{comb} che riceve come argomento due valori di tipo \texttt{int} \(n\) e \(k\) e ritorna il nuero delle combinazioni di \(n\) elementi presi \(k\) a \(k\), che a sua volta è definita come:

\begin{equation*}
comb(n, k) = \frac{n!}{k! (n - k)!}
\end{equation*}

Ecco alcuni esempi di output della funzione al variare dei parametri \(n\) e \(k\):

\begin{smlcode}
> comb(2, 2);
val it = 1: int

> comb(3, 1);
val it = 3: int

> comb(4, 2);
val it = 6: int

> comb(12, 7);
val it = 792: int

> comb(0, 0);
val it = 1: int
\end{smlcode}

\subsection*{Soluzione}

La soluzione è riportata di seguito è presente sul set di slide \texttt{en_extra_1} presenti su %
\href{https://drive.google.com/drive/u/1/folders/1mRM8Vjdcvwe5yU9ZYCYf0p2Fylxr5qBL}{drive \ExternalLink}.

\begin{listing}{!h}
\smlfile{assets/codes/2018.08/comb-fun.sml}
\caption{Definizione delle funzioni \texttt{prod} e \texttt{comb} con \texttt{fun}}
\end{listing}

\subsection{Commento della soluzione}

Da notare che in questo appello, a differenza dei precedenti, non è stato specificato il tipo di dato della funzione.

L'esercizio poteva essere risolto equivalentemente tramite la tecnica del \emph{currying} e dichiarando privatamente (usando \texttt{local}) la funzione \texttt{prod}.

\begin{listing}{!h}
\smlfile{assets/codes/2018.08/comb-val-rec.sml}
\caption{Definizione delle funzioni \texttt{prod} e \texttt{comb} con \texttt{val rec}}
\end{listing}

\subsection*{Esempio di esecuzione}
