\section{Agosto 2016}

\subsection{Testo d'esame}

Si consideri una possibile implementazione degli insiemi di interi in standard ML, in cui un insieme di interi rappresentato da una funzione da \texttt{int} a \texttt{bool}:

\begin{listing}{!h}
\begin{smlcode}
type insiemediinteri = int -> bool;
\end{smlcode}
\caption[]{Definizione del tipo di dato \texttt{insiemidiinteri}}
\end{listing}

La funzione applicata ad un numero intero ritorna true se il numero appartiene all'insieme, false altrimenti. L'insieme vuoto è quindi rappresentato da una funzione che ritorna sempre false:

\begin{listing}{!h}
\begin{smlcode}
val vuoto:insiemediinteri = fn n => false;

val vuoto = fn: insiemediinteri
\end{smlcode}
\caption{Definizione della funzione \texttt{vuoto}}
\end{listing}

ed un intero può essere aggiunto ad un insieme tramite la funzione aggiungi:

\begin{listing}{!h}
\begin{smlcode}
val aggiungi = fn f:insiemediinteri => fn x:int =>
(fn n:int => if (n = x) then
							true
						else
							false
):insiemediinteri;

val aggiungi = fn: insiemediinteri -> int -> insiemediinteri
\end{smlcode}
\caption[]{Definizione della funzione \texttt{aggiungi}}
\end{listing}

\'E possibile verificare se un intero è contenuto in un insieme tramite la funzione contiene:

\begin{listing}{!h}
\begin{smlcode}
val contiene = fn f:insiemediinteri => fn n:int => f n;
\end{smlcode}
\caption[]{Definizione della funzione \texttt{contiene}}
\end{listing}

Si implementi la funzione \texttt{intersezione}, avente tipo \texttt{insiemediinteri -> insiemediinteri -> insiemediinteri}, che dati due insiemi di interi ne calcola l'intersezione.

\textbf{IMPORTANTE}: notare il tipo della funzione! Come si può intuire da tale tipo, usa la \emph{tecnica del currying} per gestire i suoi due argomenti.

\subsection{Soluzione}

\begin{listing}{!h}
\smlfile{assets/codes/2016.08/intersezione.sml}
\caption[]{Definizione della funzione \texttt{intersezione}}
\end{listing}

\subsection{Commento della soluzione}

Si può arrivare alla soluzione per passi analizzando

\begin{smlcode}
((contiene i1 n) andalso (contiene i2 n))
\end{smlcode}

L'operatore \texttt{andalso} implementa l'operazione logica \emph{and}, mentre l'operatore \texttt{orelse} implementa l'operazione logica \emph{or}.

\subsection*{Esempio di esecuzione}

TODO
