\section{Giugno 2015}

\subsection{Testo}

Come noto, un numero naturale è esprimibile in base agli assiomi di Peano usando il seguente tipo di dato:

\begin{lstlisting}[style = SML, caption = {Definizione di numero naturale tramite gli Assiomi di Peano}]
datatype naturale = zero | successivo of naturale;
\end{lstlisting}

Usando tale tipo di dato, la \sml{somma} fra numeri naturali è esprimibile come:

\begin{lstlisting}[style = SML, caption = {Definizione della funzione \sml{somma} tramite gli Assiomi di Peano}]
val rec somma = fn zero         => (fn n => n)
				| successivo a	=> (fn n => successivo (somma a n));
\end{lstlisting}

Scrivere una funzione Standard ML, chiamata \sml{prodotto}, che ha tipo \sml{naturale -> naturale -> naturale}, che calcola il prodotto di due numeri naturali. Si noti che la funzione prodotto può usare la funzione somma nella sua implementazione.

\subsection{Guida alla soluzione}

Prendiamo confidenza con il tipo di dato definito:

\begin{lstlisting}[style = SML, caption = {Dichiarazione di numeri naturali}]
> zero;
val it = zero: naturale

> successivo(successivo zero);
val it = successivo (successivo zero): naturale
\end{lstlisting}

La \sml{somma} fra numeri naturali è esprimibile in due modi, equivalenti fra loro, un modo è quello illustrato dal professore, l'altro è il seguente:

\begin{lstlisting}[style = SML, caption = {Definizione \emph{alternativa} della funzione \sml{somma} tramite gli Assiomi di Peano}]
val rec somma = fn zero			=> (fn n => n)
 				 | successivo a => (fn n => (somma a (successivo(n))));

val somma = fn: naturale -> naturale -> naturale
\end{lstlisting}

\subsection{Commento sull'implementazione dell funzione \sml{somma}}

Entrambe le definizioni di \sml{somma} sono corrette. Nella prima definizione il caso \emph{successivo a} restituisce una funzione che mappa una variabile \sml{n} nel successivo della somma di \sml{a} con \sml{n}, nella seconda definizione, invece, il caso \emph{successivo a} restituisce una funzione che mappa una variabile \sml{n} nella somma di \sml{a} con il successivo di \sml{n}.

\medskip
Il funzionamento dell'esecuzione della funzione \sml{somma} fra due numeri naturali -- definiti secondo gli Assiomi di Peano -- è la seguente:

\smallskip
bisogna togliere un valore \sml{successivo} al primo addendo affinché risulti pari al caso base (cioè zero). Questo lo si fa \textbf{o} aggiungendo un valore \sml{successivo} alla somma del primo addendo con il secondo (1\textsuperscript{a} implementazione) \textbf{o} sommando il primo addendo con il successivo del secondo addendo (2\textsuperscript{a} implementazione).

\begin{lstlisting}[style = SML, caption = {Esempio di esecuzione di \sml{somma}}]
> somma (successivo zero) (successivo (successivo zero));
val it = successivo (successivo (successivo zero)): naturale
\end{lstlisting}

La somma di 1 e 2, risulta 3.

\subsection{Soluzione}

N.B. sono state aggiunte delle parentesi per far sì che gli argomenti dati in pasto alla funzione \sml{somma} siano delle espressioni valutabili e non delle funzioni, quali sarebbero senza le parentesi.

\begin{lstlisting}[style = SML, caption={Definizione della funzione \sml{prodotto} tramite gli Assiomi di Peano}]
val rec prodotto = fn zero			=> (fn b => zero)
					| successivo(a) => (fn b => (somma b (prodotto a b)));

val prodotto = fn: naturale -> naturale -> naturale
\end{lstlisting}

\subsection{Esempio di esecuzione}

Mostriamo un esempio di esecuzione della funzione \sml{prodotto}:

\begin{lstlisting}[style = SML, caption={Esempio di esecuzione}]
datatype naturale = zero | succ of naturale;

val rec somma = fn zero		=> (fn n => n)
				 | succ a	=> (fn n => succ (somma a n));

val rec prodotto = fn zero	  => (fn b => zero)
					| succ(a) => (fn b => (somma b (prodotto a b)));

prodotto (succ (succ (succ zero))) (succ (succ (succ zero)));
\end{lstlisting}
