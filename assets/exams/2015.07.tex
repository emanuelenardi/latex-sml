\section{Luglio 2015}

\subsection{Testo}

Si consideri il seguente tipo di dato, che rappresenta una semplice espressione avente due argomenti x e y:
\lstinputlisting[
	style = SML,
	caption = {Definizione del tipo di dato \sml{espressione Lambda}}
]{2015.07/expr.sml}

dove il costruttore \sml{X} rappresenta il valore del primo argomento \sml{x} dell'espressione, %
il costruttore \sml{Y} rappresenta il valore del secondo argomento \sml{y}, %
il costruttore \sml{Avg}, che si applica ad una coppia \sml{(e1, e2)}, rappresenta la media (intera) dei valori di \sml{e1} ed \sml{e2}, %
mentre il costruttore \sml{Mul} (che ancora si applica ad una coppia \sml{(e1, e2)}) rappresenta il prodotto dei valori di due espressioni \sml{e1} ed \sml{e2}.

\medskip
Implementare una funzione Standard ML, chiamata \sml{compute}, che ha tipo \sml{Expr -> int -> int -> int}.

\medskip
Come suggerito dal nome, \sml{compute} calcola il valore dell'espressione ricevuta come primo argomento, applicandola ai valori ricevuti come secondo e terzo argomento e ritorna un intero che indica il risultato finale della valutazione.

\medskip
\textbf{IMPORTANTE}: notare il tipo della funzione! Come si può intuire da tale tipo, la funzione riceve tre argomenti usando la \emph{tecnica del currying}. \'E importante che la funzione abbia il tipo corretto (indicato qui sopra). Una funzione avente tipo diverso da \sml{Expr -> int -> int -> int} non sarà considerata corretta.

\subsection{Soluzione}

\lstinputlisting[
	style = SML,
	caption = {Definizione della funzione \sml{compute}}
]{2015.07/compute.sml}

\subsection{Commento della soluzione}

\begin{lstlisting}[style = SML, caption = {Definizione della funzione \sml{compute}}]
val rec compute = fn +++X+++           => (‡fn x => fn y‡ => ŠxŠ)
				   | +++Y+++           => (‡fn x => fn y‡ => ŠyŠ)
				   | +++Avg(e1, e2)+++ => (‡fn x => fn y‡ => (Š(compute e1 x y) + (compute e2 x y)) div 2Š)
				   | +++Mul(e1, e2)+++ => (‡fn x => fn y‡ => (Šcompute e1 x y) * (compute e2 x y)Š)

val compute = fn: +++Expr+++ -> ‡int -> int‡ -> ŠintŠ
\end{lstlisting}

Ancora una volta il problema si risolve con una funzione ricorsiva, che sfrutta la definizione del tipo di dato \sml{Expr} per arrivare alla soluzione. %
Nota come la definizione del tipo della funzione (\sml{Expr -> int -> int -> int}) risulti un ottimo suggerimento per la risoluzione del problema.

La funzione individua 4 casi particolari: \sml{X}, \sml{Y}, \sml{Avg(e1, e2)} ed \sml{Mul(e1, e2)} tutti definiti in termini del dato \sml{Expr}. %
In ognuno dei casi vengono restituite due funzioni, le quali raccolgono i dati che verranno rielaborati nell'ultimo passaggio; %
l'unico nella quale viene implementata la logica di calcolo.
