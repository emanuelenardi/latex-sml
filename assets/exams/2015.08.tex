\section{Agosto 2015}

\subsection{Testo d'esame}

Scrivere una funzione Standard ML, chiamata \texttt{elementi_pari}, che ha tipo \texttt{'a list -> 'a list}. La funzione riceve come parametro una \(\alpha\)-lista e ritorna una \(\alpha\)-lista contenente gli elementi della lista di ingresso che hanno posizione \emph{pari} (il secondo elemento, il quarto elemento, ecc\dots).

\medskip
Per esempio

\begin{smlcode}
elementi_pari [1,5,2,10]
\end{smlcode}

ritorna

\begin{smlcode}
[5,10]
\end{smlcode}

\medskip
Si noti che la funzione \texttt{elementi_pari} non deve cambiare l'ordine degli elementi della lista rispetto all'ordine della lista ricevuta come argomento (considerando l'esempio precedente, il valore ritornato deve essere \texttt{[5,10]}, non \texttt{[10,5]}).

\medskip
Si noti, inoltre, che la funzione \texttt{elementi_pari} può usare i costruttori forniti da Standard ML per le \(\alpha\)-liste, senza bisogno di definire alcun \texttt{datatype} o altro.

\subsection{Soluzione}

\begin{listing}[!h]
\smlfile{assets/codes/2015.08/elementi-pari.sml}
\caption[]{Definizione della funzione \texttt{elementi_pari}}
\end{listing}

\subsection{Commento della soluzione}

Si può arrivare alla soluzione affrontando il problema con logica ricorsiva.

Risolviamo prima i casi base: quando ci viene data una lista vuota (\texttt{[]}) o una lista che contiene un solo elemento (\texttt{[v]}), restituiamo una \(\alpha\)-lista vuota, in quanto non esistono elementi in posizione pari.

Nel caso più interessante, ossia quello in cui sono presenti uno -- o più -- elementi in una posizione pari, la lista viene letta \emph{due elementi alla volta} (che abbiamo chiamato \texttt{a} e \texttt{b}) dei quali si tiene solo il secondo (quello in posizione pari), per poi effettuare una chiamata ricorsiva della funzione sulla coda della lista, la quale è composta dalla lista passata precedentemente alla funzione \emph{privata} dei primi due elementi (\texttt{a} e \texttt{b}).

\subsection*{Esempio di esecuzione}

Mostriamo un esempio di esecuzione della funzione \texttt{elementi_pari}:

\begin{listing}[!h]
\smlfile{assets/codes/2015.08/exec.sml}
\caption[]{Esempio di esecuzione}
\end{listing}
