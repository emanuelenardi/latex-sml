\section{Giugno 2018}

\subsection{Testo}

Si scriva una funzione \sml{conta} (avente tipo \sml{''a list -> int}) che riceve come argomento una lista di \sml{''a l}.
La funzione \sml{conta} ritorna il numero di elementi della lista senza considerare i duplicati.

\medskip
Come esempio, l'invocazione

\begin{lstlisting}
conta ["uno", "uno", "uno", "uno"];
\end{lstlisting}

deve avere risultato 1;

\begin{lstlisting}
conta [1, 2, 2, 3];
\end{lstlisting}

deve avere risultato 3;

\begin{lstlisting}
conta [2, 1, 3, 2];
\end{lstlisting}

deve avere risultato 3.

\subsubsection{Soluzione}

\begin{lstlisting}[style = SML, caption = {Definizione della funzione \sml{conta}}]
val rec conta = fn [] 	   => 0
		         | a::l    => if (List.exists (fn n => n = a) l) then
		 					      (conta l)
							  else
					              1 + conta l;

val conta = fn: ''a list -> int
\end{lstlisting}

\subsection{Commento della soluzione}

Da notare che negli esami passati non era stata usata nessuna funzione di libraria (come \sml{List.exists}) in quanto il Professor Abeni non permetteva di utilizzarle.
