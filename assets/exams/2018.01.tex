\section{Gennaio 2018}

\subsection*{Testo}

Si consideri il tipo di dato

\begin{lstlisting}
datatype lambda_expr = Var of string
					 | Lambda of string * lambda_expr
					 | Apply of lambda_expr * lambda_expr;
\end{lstlisting}

che rappresenta un'espressione del Lambda-calcolo.

Il costruttore \sml{Var} crea un'espressione costituita da un'unica funzione / variabile (il cui nome e' un valore di tipo \sml{string});
il costruttore \sml{Lambda} crea una Lambda-espressione a partire da un'altra espressione, legandone una variabile (indicata da un valore di tipo string);
il costruttore \sml{Apply} crea un'espressione data dall'applicazione di un'espressione ad un'altra.

Si scriva una funzione \sml{is_bound} (avente tipo \sml{string -> lambda_expr -> bool}) che riceve come argomenti una stringa (che rappresenta il nome di una variabile / funzione) ed una Lambda-espressione,
ritornando \sml{true} se la variabile indicata è legata nell'espressione, \sml{false} altrimenti.

\medskip
Come esempio, l'invocazione

\begin{lstlisting}
is_bound "a" (Var "a")
\end{lstlisting}

deve avere risultato \sml{false}, l'invocazione

\begin{lstlisting}
is_bound "b" (Var "a")
\end{lstlisting}

deve avere risultato \sml{false}, l'invocazione

\begin{lstlisting}
is_bound "a" (Lambda ("a", Apply((Var "a"), Var "b")))
\end{lstlisting}

deve avere risultato \sml{true}, l'invocazione

\begin{lstlisting}
is_bound "b" (Lambda ("a", Apply((Var "a"), Var "b")))
\end{lstlisting}

deve avere risultato \sml{false} e così via.

\medskip
\textbf{IMPORTANTE}: notare il tipo della funzione! La funzione usa la tecnica del currying per gestire i due argomenti.

La funzione \sml{is_bound} deve essere definita in un file .sml separato dalla definizione del tipo di dato \sml{lambda_expr}. Si consegni il file .sml contenente la definizione di \sml{is_bound}.

\subsection*{Soluzione}

\begin{lstlisting}[style = SML, caption = {Definizione della funzione \sml{is_bound}}]
val rec is_bound =
	fn s => fn Var v => s = v
			 | Lambda (v, e) => if (s = v) then
									true
								else
									is_bound s e
			 | Apply (e1, e2) => (is_bound s e1) orelse (is_bound s e2);

val is_bound = fn : string -> lambda_expr -> bool
\end{lstlisting}

\subsection{Commento della soluzione}

\omissis

\begin{itemize}
	\item \href{http://tpcg.io/IQJ8dn}{Esempio online \ExternalLink}
\end{itemize}
